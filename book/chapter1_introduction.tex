\section{\textbf{What} is quantum computing?}

\begin{fullwidth}

\textit{Quantum computing} sounds pretty intimidating to most people. What if I remove the ``quantum'' part, leaving only the ``computing'' part? 
Much friendlier, huh? 
The ``computing'' part is plain as simple: Give a system some input, and it will give you some output. 
Now comes the jargon, ``quantum''. The word ``quantum'' comes from \textit{Quantum Mechanics}, a branch of physics that study tiny things. 
Therefore, \textit{quantum computing} is essentially a way of using tiny particles to do calculation for us.
The quantum computer manipulates on subatomic particles to do calculations, so you can imagine it is EXTREMELY difficult to build a quantum computer.
Although there are prototypes of quantum computer as year of 2018, none of them has real application yet.
However, quantum computing is not that far away from us.
In fact, people believe it will become an critical technology in the following decades.

\end{fullwidth}

\section{\textbf{Why} quantum computing?}

\begin{marginfigure}[15\baselineskip]
    \includegraphics[width=\linewidth]{chapter1/quantum-computer-picture}
    \caption{Support structure for a D-WAVE quantum computer.}
    \label{fig:chapter1-quantum-computer-picture}
\end{marginfigure}

\footnote{\cite{Chapter1-quantum-computer-picture}}
In short, quantum computer can solve problems we can never solve with classical computer.
For example, there is an encryption algorithm we use everywhere, everyday, called RSA algorithm.
By using RSA algorithm, our passwords are encrypted during transmission and keep our money/data safe.
It is safe because it will take centuries for the most powerful super computer to decrypt the message encrypted by RSA.
But for quantum computer? This decryption process could be just a matter of seconds. 
What does that mean? If you have a quantum computer today, you would be able to hack almost all bank accounts on earth.
Of course, we don't want to build a quantum computer only good for hacking bank accounts, right?
There are a number of other applications like climate simulation, machine learning, material research, etc.
You already know how computers change the world.
Then just imagine the world will be changed again by quantum computers, and embrace the future!

\section{\textbf{How} can we try quantum computing?}

\begin{fullwidth}

As of April 2018, there are a few options for us to write quantum programs on a simulated quantum computer.
Why simulated quantum computer? Because the real quantum computer doesn't exist yet!
Furthermore, using simulated quantum computer can get our hands dirty already.
The Q\# language from Microsoft will be used as the main language for quantum programs throughout the book.

\end{fullwidth}
